\documentclass{beamer}
\usetheme{CambridgeUS}

\setbeamertemplate{caption}[numbered]{}

\usepackage{enumitem}
\usepackage{tfrupee}
\usepackage{amsmath}
\usepackage{amssymb}
\usepackage{gensymb}
\usepackage{graphicx}
\usepackage{txfonts}

\def\inputGnumericTable{}

\usepackage[latin1]{inputenc}                                 
\usepackage{color}                                            
\usepackage{array}                                            
\usepackage{longtable}                                        
\usepackage{calc}                                             
\usepackage{multirow}                                         
\usepackage{hhline}                                           
\usepackage{ifthen}
\usepackage{caption} 
\captionsetup[table]{skip=3pt}

\providecommand{\pr}[1]{\ensuremath{\Pr\left(#1\right)}}
\providecommand{\brak}[1]{\ensuremath{\left(#1\right)}}
\providecommand{\cbrak}[1]{\ensuremath{\left\{#1\right\}}}

\renewcommand{\thefigure}{\arabic{table}}
\renewcommand{\thetable}{\arabic{table}}                                     
                               
\title{AI1110 \\ Assignment 7}
\author{Gunjit Mittal \\ AI21BTECH11011}
\date{27 May 2022}

 
\begin{document}
	% The title page
	\begin{frame}
		\titlepage{}
	\end{frame}
	
	% The table of contents
	\begin{frame}{Outline}
    		\tableofcontents
	\end{frame}
	
	% The question
	\section{Question}
	\begin{frame}{Papoulis Exercise 4--35}
	\begin{block}{Poisson Theorem}
		If $n \to \infty$ and $p \to 0$ such that $np \to \lambda$ then 
		\begin{align}
			\label{poisson}
			\frac{n!}{k! (n-k)!} p^k q^{n-k} \xrightarrow[n \to \infty]{} e^{-\lambda} \frac{\lambda^k}{k!} \quad k = 0,1,2,\ldots
		\end{align}
	\end{block}
	
	Reasoning as in \eqref{poisson}, show that, if
	\begin{align}
		\label{given}
		k_1 + k_2 + k_3 = n \qquad p_1 + p_2 + p_3 = 1 \qquad k_1 p_1 \ll 1 \qquad k_2 p_2 \ll 1
	\end{align}
	then
	\begin{align}
		\frac{n!}{k_1! k_2! k_3!} \simeq \frac{n^{k_1 + k_2}}{k_1! k_2!} \qquad p_3^{k_3} \simeq e^{-n(p_1 + p_2)}
	\end{align}
	\end{frame}
	
	% The proof
	\section{Proof of the first part}
	\begin{frame}{Proving the first part}
	From \eqref{given}, we have $k_3 = n - k_1 - k_2$
	
	Thus,
	\begin{align}
		\frac{n!}{k_1! k_2! k_3!} &= \frac{n!}{k_1! k_2! (n - k_1 - k_2)!} \\
		&= \frac{n(n-1) \cdots (n - k_1 - k_2 + 1)}{k_1! k_2!} \\
		&= \frac{n(n-1) \cdots (n - k_1 - k_2 + 1)}{n^{k_1 + k_2}} \frac{n^{k_1 + k_2}}{k_1! k_2!} \\
		&= \brak{1-\frac1n} \brak{1-\frac2n} \cdots \brak{1-\frac{k_1 + k_2 - 1}{n}} \frac{n^{k_1 + k_2}}{k_1! k_2!} \\
		&= \brak{\prod_{m=0}^{k_1 + k_2 - 1} \brak{1 - \frac{m}{n}}} \frac{n^{k_1 + k_2}}{k_1! k_2!}
	\end{align}
	\end{frame}
		
	\begin{frame}
	Now, if we assume that $k_1 + k_2$ is finite, i.e.,
	\begin{align}
		\label{assumption}
		\text{as } n \to \infty, k_1 + k_2 \ll n
	\end{align}
	
	Then the finite product
	\begin{align}
		\prod_{m=0}^{k_1 + k_2 - 1} \brak{1 - \frac{m}{n}} 
	\end{align}
	tends to unity as $n \to \infty$
	
	\begin{block}{}
	Therefore,
		\begin{align}
			\frac{n!}{k_1! k_2! k_3!} \simeq \frac{n^{k_1 + k_2}}{k_1! k_2!} \qed
		\end{align}
	\end{block}
	\end{frame}

	\section{Proof of the second part}
	\begin{frame}{Proving the second part}
	From \eqref{given}, we have $p_3 = 1 - p_1 - p_2$ 
	\begin{align}
		p_3^{k_3} &= (1 - p_1 - p_2)^{k_3} 
	\end{align}
	We have already assumed that $k_1 \ll n$ and $k_2 \ll n$ in \eqref{assumption}
	
	Assume $p_1 \ll 1$ and $p_2 \ll 1$ satisfying $k_1 p_1 \ll 1$,  $k_2 p_2 \ll 1$
	\begin{align}
		\therefore p_1 + p_2 &\ll 1
	\end{align}
	 
	By the definition of $e^x$,
	\begin{align}
		e^x &= \lim_{n \to \infty} \brak{1 + \frac{x}{n}}^n \\
		&= \lim_{n \to \infty} \sum_{k=0}^n \frac{n!}{k! (n-k)!} \brak{\frac{x}{n}}^k \\
		&= \lim_{n \to \infty} \sum_{k=0}^n \frac{n (n-1) \cdots (n-k+1)}{n^k} \brak{\frac{x^k}{k!}}  \\
	\end{align}
	\end{frame}
	
	\begin{frame}
	\begin{align}
		e^x &= \sum_{k=0}^{\infty} \brak{\brak{\frac{x^k}{k!}} \lim_{n \to \infty} \brak{1-\frac1n} \brak{1-\frac2n} \cdots \brak{1 - \frac{k-1}{n}}} \\
		&= \sum_{k=0}^{\infty} \frac{x^k}{k!} \\
		&= 1 + x + \frac{x^2}{2!} + \cdots \\
		\implies e^{-x} &= 1 - x + \frac{x^2}{2!} + \cdots
	\end{align}
	
	For $x \ll 1$, the higher-order terms can be neglected
	\begin{align}
		e^{-x} &\simeq 1 - x \\
		\implies 1 - p_1 - p_2 &\simeq e^{-(p_1 + p_2)} \\
		\implies (1 - p_1 - p_2)^{k_3} &\simeq e^{-k_3(p_1 + p_2)}
	\end{align}
	\end{frame}
	
	\begin{frame}
	\begin{align}
		\therefore p_3^{k_3} \simeq e^{-k_3(p_1 + p_2)}
	\end{align}
	Also, $k_1 + k_2 \ll n \implies k_3 \simeq n$ since $k_1 + k_2 + k_3 = n$
	
	\begin{block}{}
	Therefore,
		\begin{align}
			p_3^{k_3} \simeq e^{-n(p_1 + p_2)} \qed
		\end{align}
	\end{block}
	\end{frame}
	
	% The follow-up
	\section{Follow-up}
	\begin{frame}{Follow-up}
	\begin{block}{Random Poisson points in non-overlapping intervals}
	Use these results to justify, for non-overlapping intervals $t_a$ and $t_b$,
		\begin{align}
			\pr{k_a~\mathrm{in}~t_a, k_b~\mathrm{in}~t_b} = e^{-\lambda t_a} \frac{(\lambda t_a)^{k_a}}{k_a!} e^{-\lambda t_b} \frac{(\lambda t_b)^{k_b}}{k_b!}
		\end{align}
		where $\pr{k~\mathrm{in}~t}$ denotes the probability that $k$ of $n$ randomly placed points in the interval $\brak{-\frac{T}{2}, \frac{T}{2}}$ will lie in an interval of length $t$ and $\lambda = \frac{n}{T}$ is constant
	\end{block}
	\end{frame}
	
	% The justification to the follow-up
	\section{Justification}
	\begin{frame}{Justification}
	Let $k_1 = k_a$, $k_2 = k_b$ and $k_3 = n - k_a - k_b$ denote the number of points lying in $t_a$, $t_b$ and outside both $t_a$ and $t_b$ respectively out of a total of $n$ randomly placed points in $\brak{-\frac{T}{2}, \frac{T}{2}}$
	\begin{align}
		k_1 + k_2 + k_3 = n
	\end{align}
	
	Let $p_1$, $p_2$ and $p_3$ denote the probabilities that an arbitrary point lies in $t_a$, $t_b$ and outside both $t_a$ and $t_b$ respectively
	\begin{align}
		p_1 + p_2 + p_3 = 1
	\end{align}
	
	Assuming small intervals, i.e., $t_a, t_b \ll T$
	\begin{align}
		k_1 p_1 \ll 1 \qquad k_2 p_2 \ll 1
	\end{align}
	\end{frame}		
	
	\begin{frame}
	The conditions have thus been met and we can now use the previously proven results
	\begin{align}
		\pr{k_a~\mathrm{in}~t_a, k_b~\mathrm{in}~t_b} &= \frac{n!}{k_a! k_b! (n-k_a-k_b)!} p_1^{k_a} p_2^{k_b} p_3^{n - k_a - k_b} \\
		&= \frac{n!}{k_1! k_2! k_3!} p_1^{k_1} p_2^{k_2} p_3^{k_3} \\
		&\simeq \frac{n^{k_1 + k_2}}{k_1! k_2!} p_1^{k_1} p_2^{k_2} e^{-n(p_1 + p_2)} \\
		&= e^{-np_1} \frac{(np_1)^{k_1}}{k_1!} e^{-np_2} \frac{(np_2)^{k_2}}{k_2!}
	\end{align}
	
	Now, $p_1 = \frac{t_a}{T} \implies np_1 = \frac{n}{T} t_a = \lambda t_a$
	
	Similarly, $np_2 = \lambda t_b$
	
	Substituting back, we get the desired result.
	\end{frame}
	
\end{document}